\section*{Flink Memory Management}
In order to achieve memory optimization, Flink uses off-heap memory.
This is explained in detail in the JVM. If you are interested, you can
check it out. Using off-heap memory can reduce the limitation of
allocated memory (time and space), and at the same time can reduce
memory GC. Using off-heap memory can greatly reduce the heap memory
(only the Remaining Heap needs to be allocated), so that the
TaskManager can be expanded to hundreds of GB of memory is not a
problem. In the IO operation, the use of off-heap memory can be
zero-copy, and the use of on-heap memory must be copied at least once.
It should be noted that off-heap memory is shared between processes.

\subsection*{Memory Usage Classification}
The basic idea of Flink's memory is to use an abstract memory pool to
pre-allocate a large amount of memory to increase the proportion of
reuse. Standardize the size of memory to improve efficiency. Memory is
mainly divided into the following three situations:

\paragraph{Memory Pool:} 
The memory management pool is a massive collection composed of many
Memory Segments. The algorithm applied in FLINK, such as sorting,
cleaning data, etc., will apply for distribution of them and store
their serialized data inside. This memory can be reused and allocated.
Under the normal circumstances, the occupied memory usage of this part
of the application accounts for the majority of entire memory.

\paragraph{Network Buffer:} 
Big data data communication will certainly occupy a considerable
amount of memory. In Flink's network buffer, there is a pre-allocated
32K buffer, which is similar to the IO buffer in the network cable
driver. It will be allocated when taskmanager starts. The default
number is 2K, which can be configured through
taskmanager.network.numberOfBuffers.

\paragraph{User}
Memory used by users This part of memory is mainly used for some basic
data structures (such as taskmanager) for users and some code-related
memory usage.

\subsection*{Abstract Data Structure of Memory}
The memory segment of Memory Segment can be considered as a part of
heap memory in JVM.
\begin{lstlisting}
	public abstract class HybridMemorySegment extends MemorySegment {

	}
\end{lstlisting}

\begin{lstlisting}
	public abstract class MemorySegment {

	}
\end{lstlisting}


\subsection*{Memory Mapping and Application}

\paragraph{Binary Processing}
Binary processing is actually a processing method of the following
serialization, through which a binary data stream can be obtained.
Then push the data stream to the corresponding operator, and operate
the data through it, such as Sort, Join, etc.

\paragraph{Memory Mangement of Scala}
Flink fully supports all Scala data types

\paragraph{Serialization}
Serialization is actually easy to understand. It is the process of
converting binary streams and data structures back and forth, or
encoding and decoding.  Flink implements its own serialization
framework without using other ready-made frameworks. This also has a
lot to do with the single data structure of Flink itself. Flink
obtains type information (TypeInformation) through JAVA's reflection
mechanism, which mainly contains the following types:

\textbf{BasicTypeInfo:} Any Java basic type (boxed) or String

\textbf{BasicArrayTypeInfo:} Any Java basic type array (boxed) or
String array

\textbf{WritableTypeInfo:} The implementation class of any Hadoop
Writable interface.

\textbf{TupleTypeInfo:} Any Flink Tuple type (support Tuple1 to
Tuple25). Flink tuples are Java Tuple implementations of fixed length
and fixed type.

\textbf{CaseClassTypeInfo:} Any Scala CaseClass (including Scala
tuples).

\textbf{PojoTypeInfo:} Any POJO (Java or Scala), for example, all
member variables of a Java object are either defined by public
modifiers or have getter/setter methods.

\textbf{GenericTypeInfo:} Any class that cannot match the previous
types. Flink can serialize basic data types through TypeSerializer.
For GenericTypeInfo, Kryo is used to achieve serialization.

\subsection*{Memory Management (memory pool)}
MemoryManager provides two internal classes:%

\begin{itemize}
	\item{HybridHeapMemoryPool: Byte array is generally used}

	\item{HybridOffHeapMemoryPool: ByteBuffer is generally allocated} 
\end{itemize}

\section*{Task Memory Manager}
\begin{center}
	\begin{tabular}{|l|l|}
		\hline
		Config. Option & Description \\
		\hline
		taskmanager.memory.size		& Sets the size managed by task memory
									  manager (sorting, hashing,
									  caching).  The default is 0. \\
		\hline
		taskmanager.memory.size		& Value is less than or equall to 0.
									  It will be allocated according
									  to the
									  taskmanager.memory.fraction
									  which is 0.7 \\
		\hline
		taskmanager.memory.off-heap & The maximimum value of
									  taskmanager is
									  taskmanager.heap.size=networkBuffMB-offHeapSize
	\end{tabular}
\end{center}
